\documentclass[tikz]{standalone}
\usepackage{tikz-3dplot}
\usetikzlibrary{angles,arrows,backgrounds}
\tikzstyle{background grid}=[draw, step=2mm, gray, very thin]

% Colors
\colorlet{fillcol}{black!25}     % fill color
\colorlet{veccol}{black!60}      % input vectors
% Reference frame axis styles
\tikzstyle{vec}=[-stealth,veccol,scale=2,line cap=round]
\tikzstyle{plane}=[very thin,fill=fillcol,draw=veccol,line cap=round]
\tikzstyle{rightangle}=[draw=veccol,very thin,angle radius=4mm,
pic text=.,pic text options={veccol},anchor=west]

\begin{document}

\tdplotsetmaincoords{70}{135}
\begin{tikzpicture}[tdplot_main_coords, scale=1.5, % show background grid,
  every node/.style={scale=0.4}, every circle/.style={radius=0.25pt}]
  % Macros for key quantities
  \pgfmathsetmacro{\psiangle}{50}
  \pgfmathsetmacro{\qlen}{0.5}
  \pgfmathsetmacro{\px}{1}     % P2 x component
  \pgfmathsetmacro{\py}{0}     % P2 y component
  \pgfmathsetmacro{\pz}{0}     % P2 z component
  % Coordinates
  \coordinate (O) at (0,0,0);
  \coordinate (N) at (0,0,\qlen);
  \coordinate (P1) at (1,0,0);

  % Create rotated frame and rotated vector
  \tdplotsetrotatedcoords{\psiangle}{0}{0}
  % Get coordinates of P2 in main coordinate system
  \tdplottransformrotmain{\px}{\py}{\pz}
  \coordinate (P2) at (\tdplotresx,\tdplotresy,\tdplotresz);
  % Fill polygon for plane
  \filldraw[tdplot_main_coords,plane]
  (O) -- (P1) -- (P2) -- cycle;
  % Main frame
  \draw[vec,black] (O) -- (N) node[anchor=east]{\(n\)};
  % Points
  \filldraw[vec] (O) circle node [anchor=west]{\(P_{0}\)}
  (P1) circle node [anchor=east]{\(P_{1}\)}
  (P2) circle node [anchor=west]{\(P_{2}\)};
  \coordinate (P3) at (barycentric cs:O=1,P1=0.3,P2=0.4);
  % Right angle indicator
  \path (N) -- (O) -- (P3)
  pic [rightangle] {right angle=N--O--P3};
\end{tikzpicture}

\end{document}


%%% Local Variables:
%%% mode: latex
%%% TeX-master: t
%%% End:
